\documentclass[letterpaper,12pt]{article}

\usepackage{fontspec,xltxtra,xunicode}      %可使用系统自带字体
%\usepackage{}

\usepackage{setspace}    %使用行间距宏包
\usepackage{geometry}
\usepackage{graphicx}
\usepackage{amssymb}
\usepackage{indentfirst}
\usepackage{enumerate}
\usepackage{amsmath}
\usepackage{abstract}
\usepackage[slantfont, boldfont]{xeCJK}
\usepackage[colorlinks,linkcolor=black,anchorcolor=black,citecolor=black]{hyperref}
\usepackage{titlesec}
\usepackage{latexsym}
\usepackage{amsbsy}
\usepackage{amsthm}
\usepackage{amsfonts}
\usepackage{tocvsec2}
\usepackage{longtable}
\usepackage{booktabs}                % 用于表格中加下划线
\usepackage{fancyhdr}                % 页眉页脚
\usepackage{makeidx}                 % 建立索引
\usepackage{bbding}                  % 一些特殊符号
\usepackage{cite}                    % 支持引用
\usepackage{multirow}                %使用多栏宏包
\setlength{\skip\footins}{0.5cm}     % 脚注与正文的距离

\usepackage{listings}



\newcommand{\upcite}[1]{\textsuperscript{\textsuperscript{\cite{#1}}}}  %设置上标引用
\geometry{top=1.2in,bottom=1.2in,left=1.2in,right=1in}  %设置页边距
\defaultfontfeatures{Mapping=tex-text}
\XeTeXlinebreaklocale "zh"
\XeTeXlinebreakskip = 0pt plus 1pt minus 0.1pt   %设置文章内自动换行
%\setlength{\parindent}{2em}  %设定首行缩进

%中文字体设置
\setCJKmainfont[BoldFont=Adobe Heiti Std,ItalicFont=Adobe Kaiti Std]{Adobe Song Std}
\setCJKsansfont{Adobe Heiti Std}
\setCJKmonofont{Adobe Fangsong Std}
 
\setCJKfamilyfont{zhsong}{Adobe Song Std}
\setCJKfamilyfont{zhhei}{Adobe Heiti Std}
\setCJKfamilyfont{zhfs}{Adobe Fangsong Std}
\setCJKfamilyfont{zhkai}{Adobe Kaiti Std}
\setCJKfamilyfont{zhli}{LiSu}
\setCJKfamilyfont{zhyou}{YouYuan}
 
\newcommand*{\songti}{\CJKfamily{zhsong}} % 宋体
\newcommand*{\heiti}{\CJKfamily{zhhei}}   % 黑体
\newcommand*{\kaishu}{\CJKfamily{zhkai}}  % 楷书
\newcommand*{\fangsong}{\CJKfamily{zhfs}} % 仿宋
\newcommand*{\lishu}{\CJKfamily{zhli}}    % 隶书
\newcommand*{\youyuan}{\CJKfamily{zhyou}} % 幼圆

%英文字体设置
\setmainfont{Times New Roman}
\setsansfont{Arial}
\setmonofont{Consolas}

%设置字体大小
\newcommand{\chuhao}{\fontsize{42pt}{\baselineskip}\selectfont}      %初号字体
\newcommand{\xiaochu}{\fontsize{36pt}{\baselineskip}\selectfont}  %小初号字体
\newcommand{\yihao}{\fontsize{28pt}{\baselineskip}\selectfont}       %一号字体
\newcommand{\erhao}{\fontsize{21pt}{\baselineskip}\selectfont}       %二号字体
\newcommand{\xiaoer}{\fontsize{18pt}{\baselineskip}\selectfont}   %小二号字体
\newcommand{\sanhao}{\fontsize{15.75pt}{\baselineskip}\selectfont}   %三号字体
\newcommand{\sihao}{\fontsize{14pt}{\baselineskip}\selectfont}       %四号字体
\newcommand{\xiaosi}{\fontsize{12pt}{\baselineskip}\selectfont}   %小四号字体
\newcommand{\wuhao}{\fontsize{10.5pt}{\baselineskip}\selectfont}     %五号字体
\newcommand{\xiaowu}{\fontsize{9pt}{\baselineskip}\selectfont}    %小五号字体
\newcommand{\liuhao}{\fontsize{7.875pt}{\baselineskip}\selectfont}   %六号字体
\newcommand{\qihao}{\fontsize{5.25pt}{\baselineskip}\selectfont}     %七号字体

%更新目录指令
\renewcommand{\contentsname}{ \centerline {\heiti \sanhao{目\quad 录}}}
\renewcommand{\refname}{\centerline {\heiti \xiaosi{参考文献}}}


\lstset{ %  
extendedchars=false,            % Shutdown no-ASCII compatible  
language=Matlab,                % choose the language of the code  
basicstyle=\footnotesize\tt,    % the size of the fonts that are used for the code  
tabsize=3,                            % sets default tabsize to 3 spaces  
numbers=left,                   % where to put the line-numbers  
numberstyle=\tiny,              % the size of the fonts that are used for the line-numbers  
stepnumber=1,                   % the step between two line-numbers. If it's 1 each line  
                                % will be numbered  
numbersep=5pt,                  % how far the line-numbers are from the code   %  
keywordstyle=\color[rgb]{0,0,1},                % keywords  
commentstyle=\color[rgb]{0.133,0.545,0.133},    % comments  
stringstyle=\color[rgb]{0.627,0.126,0.941},      % strings  
backgroundcolor=\color{white}, % choose the background color. You must add \usepackage{color}  
showspaces=false,               % show spaces adding particular underscores  
showstringspaces=false,         % underline spaces within strings  
showtabs=false,                 % show tabs within strings adding particular underscores  
frame=single,                 % adds a frame around the code  
captionpos=b,                   % sets the caption-position to bottom  
breaklines=true,                % sets automatic line breaking  
breakatwhitespace=false,        % sets if automatic breaks should only happen at whitespace  
title=\lstname,                 % show the filename of files included with \lstinputlisting;  
                                % also try caption instead of title  
mathescape=true,escapechar=?    % escape to latex with ?..?  
escapeinside={\%*}{*)},         % if you want to add a comment within your code  
%columns=fixed,                  % nice spacing  
%morestring=[m]',                % strings  
%morekeywords={%,...},%          % if you want to add more keywords to the set  
%    break,case,catch,continue,elseif,else,end,for,function,global,%  
%    if,otherwise,persistent,return,switch,try,while,...},%  
}  




\title{Homework 3 Report}
\author{Peide Li}

\begin{document}
\maketitle

In this experiment, I used Matlab to solve regression problems with $l_{2}$-norm regularization. The original problem can be described in this way. Given training data set, we need to find a linear model which can minimize the error of prediction. In order to avoid over-fitting problems, we add regularization term to penalize the error if the norm of parameters is very large. Assume $X$ is the designed matrix, $y$ is the target vector in our training set, $\lambda$ is the regularization term. We need to solve the optimization problem:
\begin{center}
$E(w) = \underset{w}{min} \frac{1}{2}\left \| Xw - y \right \|^{2}_{2} + \frac{\lambda}{2}\left \| w \right \|^{2}_{2}$
\end{center}
By setting the derivative $\frac{\partial E(w)}{\partial w} = 0$, we can get the least square estimate of the parameter:
\begin{center}
 $w = \left ( X^{T}X + \lambda I \right )^{-1} X^{T}y$
\end{center}
Then I use this formula to code the ridge regression solver in Matlab. The function is shown below. It needs three inputs: designed matrix, target vector, and regularization parameter. It will give out the $w$ which can minimize the loss function $E(w)$.

\begin{lstlisting}  
%%%%%%%%%%%%%%%%%%%%%%%%%%%%%%%%%%%%%%%%%%%%%%%%%%%%%%%%%%%%%%%%%%%
%%% Ridge Regression Function
%%% Input : X(predictors), Y(response), lambda(regularizaiotn parameter)
%%% Output: w (renewed parameters)

function [w] = RidgeRegress(X, Y, lambda)
[a, b] = size(X' * X);

%calculate the invers part
den = X' * X + eye(a,b) * lambda;

if det(den) == 0  %Prevent that the projection matrix is not inversible 
    disp('The matrix is not inversible');
    w = 0;
else
    w = den^(-1) * X' * Y;
end

end
\end{lstlisting}  

Next, I use the ridge regression solver to get the least square estimate of the parameters on the Diabetes dataset. The regularization parameter vary from $1e-5, 1e-4, 1e-3, 1e-2, 1e-1, 1, 10$. And I use mean squared error to measure the error. The Figure 1 shows the MSE against the value of the log $\lambda$.
\begin{center}
\begin{figure}
\includegraphics[width = 16cm, height = 10cm]{"error.jpg"}
\caption{MSE against $log \lambda$}
\end{figure}
\end{center}
Finally, I performed 5-fold cross validation on the training data to find the best $\lambda$ for the model. I divided the training set almost evenly into five groups, NO.1-50, 51-100, 101-150, 151-200, 201-242. So there are five different combinations of training set and cross validation set. For each value of $\lambda$, I use these sets to train the models and computed their cross validation errors(just the same as the MSE). I found the $\lambda = 10$ has the minimum cross validation error $1.4044 \times 10 ^{5}$. So I set $\lambda = 10$ and use the whole training set to train the model again. The best $\lambda$ obtained from cross validation procedure is also pointed out in the Figure 1.

\quad \\


The PDF report and the original Matlab code can be found at my github site:   
\url{https://github.com/Kira233767/HomeWork-3.git}


\end{document}
